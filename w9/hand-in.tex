\documentclass{article}
\usepackage[left=1in,right=1in,bottom=0.8in,top=0.8in]{geometry}
\geometry{a4paper}
\usepackage[parfill]{parskip}
\usepackage{graphicx}
\usepackage{xcolor}
\usepackage{hyperref}
\usepackage{amsmath,amssymb}
\usepackage{listings}
\usepackage[outputdir=build]{minted}

\title{Cryptology Exercise Week 9}
\author{Zijun Yu 202203581}
\date{Octobor 2023}

\begin{document}

\maketitle

\section*{CPA security of El Gamal}

The adversary $Adv$ plays the CPA game, which means that it has the public keys
$G$, $\alpha$, and $\alpha^a$. It then chooses a message $m$ and sends to the oracle,
and gets back $(\alpha^r, x)$. Let $G$, $\alpha$, $\alpha^a$, $\alpha^r$, and $x\cdot m^{-1}$ be
the inputs to $B$, namely $G$, $\alpha$, $\alpha^a$, $\alpha^b$, and $\alpha^c$. So $r=b$.
Then notice that $x$ is either $\alpha^{ab}\cdot m$ when we are in the real world,
or $\alpha^{ab}\cdot s$, for some random $s$, when we are in the ideal world. We can rewrite
$\alpha^{ab} \cdot s = \alpha^{ab} \cdot m \cdot \alpha^{t} = \alpha^{ab+t} \cdot m$.
Because $s$ is uniformly random, $ab+t$ is also uniformly random. This means that the oracle
is always responding with $x = \alpha^{c}\cdot m$, where $c$ is either $ab$ or uniformly random,
and $Adv$ can distinguish between these two cases with advantage $\epsilon$.
We then simply let $B$ output the same as $Adv$ does and $B$ will have $\epsilon$ advantage
at distinguishing $c$ is $ab$ or uniformly random in $\alpha^{c}$ (since $x\cdot m^{-1}$ is exactly $\alpha^{c}$).
\end{document}
