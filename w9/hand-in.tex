\documentclass{article}
\usepackage[left=1in,right=1in,bottom=0.8in,top=0.8in]{geometry}
\geometry{a4paper}
\usepackage[parfill]{parskip}
\usepackage{graphicx}
\usepackage{xcolor}
\usepackage{hyperref}
\usepackage{amsmath,amssymb}
\usepackage{listings}
\usepackage[outputdir=build]{minted}

\title{Cryptology Exercise Week 9}
\author{Zijun Yu 202203581}
\date{Octobor 2023}

\begin{document}

\maketitle

\section*{CPA security of El Gamal}

The algorithm $B$ works by sending $(\alpha^b, \alpha^c \cdot m)$ to $Adv$ and outputs
the same result as A does.

In the case where $B$ is called on $(\alpha^a, \alpha^b, \alpha^c)$ where $c = ab$,
what $Adv$ sees is identical to interacting with the “real” oracle. In the case where $B$
is called on $(\alpha^a, \alpha^b, \alpha^c)$ where $c$ is random, what $Adv$ sees
is that it is talking to the "ideal" oracle, since $\alpha^c \cdot m$ is indistinguishable
from the encryption of a random message, which is what the "ideal" oracle does,
i.e. $\alpha^{ab} \cdot r$. (Given $c$ and $r$ are uniformly random values
from the group, $\alpha^c \cdot m$ and $\alpha^{ab} \cdot r$ are also two uniformly random values,
hence they are indistinguishable.)

This construction thus turns an adversary that breaks El-Gamal into one that breaks DDH with the same
advantage.

\end{document}
