\documentclass{article}
\usepackage{graphicx} % Required for inserting images
\usepackage{listings}
\usepackage[outputdir=build]{minted}
\usepackage{amssymb,amsmath}
\usepackage[margin=1in]{geometry}

\title{Cryptology Exercise Week 1}
\author{Zijun Yu 202203581}
\date{September 2023}

\begin{document}

\maketitle

We first calculate the frequency of each letter in Python. Part of the output looks like this:

\begin{minted}{python}
('C', 32), ('B', 21), ('K', 20), ('P', 20),
\end{minted}

The 4 most frequent English letters are E, T, A, and O. Let's try matching C with E and B with T.
According to the index of the letters, We get:

\begin{align*}
    4a + b  & \equiv 2 \pmod{26} \\
    19a + b & \equiv 1 \pmod{26}
\end{align*}

Solving the equations yields $a = 19, b = 4$. So the encryption function is $f(x) = 19 x + 4 \pmod{26}$. And the corresponding
decryption function is $f^{-1}(x) = 11 * (x - 4) \pmod{26}$, where 11 is the inverse of 19 modulo 26.

By applying the decryption function to the ciphertext, we get the plaintext:

\begin{minted}{text}
OCANADATERREDENOSAIEUXTON...
\end{minted}

And ChatGPT translated this text without spaces to the following English version:

\begin{minted}{text}
Canada, land of our ancestors, thy brow is wreathed with glorious jewels.
In thy armoured embrace, both the sword and the cross thou dost bear. Thy
history is an epic of the most brilliant exploits, and thy valour steeped
in faith will protect our homes and our rights.
\end{minted}
\end{document}
