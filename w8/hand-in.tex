\documentclass{article}
\usepackage[left=1in,right=1in,bottom=0.8in,top=0.8in]{geometry}
\geometry{a4paper}
\usepackage[parfill]{parskip}
\usepackage{graphicx}
\usepackage{xcolor}
\usepackage{hyperref}
\usepackage{amsmath,amssymb}
\usepackage{listings}
\usepackage[outputdir=build]{minted}

\title{Cryptology Exercise Week 8}
\author{Zijun Yu 202203581}
\date{Octobor 2023}

\begin{document}

\maketitle

\section*{Repeated Squaring}

\subsection*{Question 1}
We prove by induction that the algorithm is correct. First, at the end of the first iteration
where $i=k-1$, we have
$$x = (1^2 \text{ mod } n) \cdot a^{z_{k-1}} \text{ mod } n = a^{z_{k-1}} = a^{Z_i}$$
Then suppose at the begining of some n'th iteration, $x=a^{Z_i}=a^{z_{k-1}z_{k-2}...z_{i}}$ holds.
At the end of this iteration, we have
\begin{equation*}
    \begin{split}
        x &= \left(a^{z_{k-1}z_{k-2}...z_{i}}\right)^2 \cdot a^{z_{i-1}} \text{ mod } n \\
        &= a^{z_{k-1}z_{k-2}...z_{i}0} \cdot a^{z_{i-1}} \text{ mod } n \\
        &= a^{z_{k-1}z_{k-2}...z_{i}z_{i-1}} \text{ mod } n \\
        &= a^{Z_{i-1}} \text{ mod } n
    \end{split}
\end{equation*}
Hence by induction, we have $x=a^{Z_0}=a^z$ at the end of the algorithm.

\subsection*{Question 2}
In each iteration, we have to compute $x^2 \text{ mod } n$.
On average, half of the $k$ bits are 1, so in half of the iterations, we have to compute
$x \cdot a^{z_i} \text{ mod } n$. Therefore, the expected number of multiplications is
$$(1+\frac{1}{2}) \cdot k = k + \frac{1}{2}k$$

\subsection*{Question 3}
\begin{enumerate}
    \item $x:=1$
    \item For $i=k-1, k-3, ... ,0$, do
          \begin{enumerate}
              \item $x := x^2 \text{ mod } n$
              \item $x := x^2 \text{ mod } n$
              \item $x := x \cdot a^{z_i z_{i-1}} \text{ mod } n$
                    ($a^{z_i z_{i-1}}$ is either $a$ or the $a^2$ or $a^3$ we have computed beforehand, this step is empty if $z_i z_{i-1}=00$)
          \end{enumerate}
    \item Return x
\end{enumerate}
At the last iteration, if we have less than 2 bits left, we do the two steps from the original algorithm.

The two square operations will append two 0's to the exponent of $a^{z_k z_{k-1} ... z_i}$, so it is trivial
to see that the algorithm is correct by the same induction proof as in Question 1.

In each iteration, the possibility that we can skip $x \cdot a^{z_i z_{i-1}}$ is $1/4$.
Assume that 2 devides $k$, we then have exactly $k/2$ iterations. So the expected number of multiplications is
$$(2 + \frac{3}{4}) \cdot \frac{k}{2} + 2 = \frac{11}{8}k + 2$$
As long as $k$ is reasonly big, which it is, the new algorithm is faster.

\subsection*{Question 4}

If we scan $n$ bits at once, the expected number of multiplications, assuming that $n$ divides $k$, is
\begin{equation*}
    \begin{split}
        &(n + \frac{2^n-1}{2^n}) \cdot \frac{k}{n} + 2^n - 2 \\
        =& k + \frac{2^n - 1}{n 2^n}k + 2^n - 2
    \end{split}
\end{equation*}
Where $2^n - 2$ comes from computing $a^2$, $a^3$, ..., $a^{2^n-1}$.
With a fixed $k$, the number first goes small as $n$ increases, but then goes up again as $n$ increases.
And with $k$ increasing, the optimal choice of $n$ is also increasing.
For example, if $k=100$, the optimal $n$ is 3, and if $k=1000$, the optimal $n$ is 5.

\end{document}
