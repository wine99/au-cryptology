\documentclass{article}
\usepackage[left=1in,right=1in,bottom=0.8in,top=0.8in]{geometry}
\geometry{a4paper}
\usepackage[parfill]{parskip}
\usepackage{graphicx}
\usepackage{xcolor}
\usepackage{hyperref}
\usepackage{amsmath,amssymb}
\usepackage{listings}
\usepackage[outputdir=build]{minted}

\title{Cryptology Exercise Week 10}
\author{Zijun Yu 202203581}
\date{Octobor 2023}

\begin{document}

\maketitle

\section*{Correctness of LWE-based encryption}

We have the ciphtertext $(\mathbf{u}, v)$, where
$$u=b_1\mathbf{a_1} + b_2\mathbf{a_2} + ... + b_m\mathbf{a_m}$$
and
$$v=b_1(\mathbf{a_1} \cdot \mathbf{s} + e_1) + b_2(\mathbf{a_2} \cdot \mathbf{s} + e_2) + ... + b_m(\mathbf{a_m} \cdot \mathbf{s} + e_m) + \lceil q/2 \rceil w$$
The decryption is $v - \mathbf{s} \cdot \mathbf{u}$, which is
$$b_1e_1 + b_2e_2 + ... + b_me_m + \lceil q/2 \rceil w$$
Because $\sum_{i=1}^{m}|e_i| < q/4 -1$, we have
$$|b_1e_1 + b_2e_2 + ... + b_me_m| <= \sum_{i=1}^{m}|e_i| < q/4 -1$$
So in the case of $w=0$, the ciphtertext is $b_1e_1 + b_2e_2 + ... + b_me_m$, of which the absolute value is less than $q/4 -1$,
hence it is closer to 0 than to $\lceil q/2 \rceil$, so the decryption is 0. In the case of $w=1$, the ciphtertext is
$\lceil q/2 \rceil - X$ where $|X| < q/4 -1$, hence it is closer to $\lceil q/2 \rceil$ than to 0, so the decryption is 1.

\end{document}
