\documentclass{article}
\usepackage[left=1in,right=1in,bottom=0.8in,top=0.8in]{geometry}
\geometry{a4paper}
\usepackage[parfill]{parskip}
\usepackage{graphicx}
\usepackage{xcolor}
\usepackage{hyperref}
\usepackage{amsmath,amssymb}
\usepackage{listings}
\usepackage[outputdir=build]{minted}

\title{Cryptology Exercise Week 13}
\author{Zijun Yu 202203581}
\date{December 2023}

\begin{document}

\maketitle

\section*{Misuse of randomness in Schnorr signatures}

Since the verifier can compute $c$ himself and $c = \alpha^r$, the verifier can simply tell
if $r_1$ and $r_2$ are the same by comparing $c_1$ and $c_2$. This follows from the fact that $r_1$ and $r_2$ are chosen in $\mathbb{Z}_q$
which is the order of $\alpha$ and thus $c_1 = c_2$ if and only if $r_1 = r_2$.

When the same $r$ is used in two signatures, the verifier can compute $z_1 - z_2 = r + e_1 a - (r + e_2 a) = (e_1 - e_2) a \mod q$.
The verifier can then simply compute the secret key $a$ by $a = (z_1 - z_2) (e_1 - e_2)^{-1} \mod q$.

For the second senario, the verifier can still compute $z_1 - z_2 = r_1 + e_1 a - (r_2 + e_2 a) = (e_1 - e_2) a + (i - j) u \mod q$
and find the secret key by $a = (z_1 - z_2 - (i - j) u) (e_1 - e_2)^{-1} \mod q$.

\end{document}
