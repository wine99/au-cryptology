\documentclass{article}
\usepackage{graphicx} % Required for inserting images
\usepackage{listings}
\usepackage[outputdir=build]{minted}
\usepackage{amssymb,amsmath}
\usepackage[margin=1.2in]{geometry}

\title{Cryptology Exercise Week 5}
\author{Zijun Yu 202203581}
\date{Octobor 2023}

\begin{document}

\maketitle

\section*{Exercise 6.6}

According to the definition of AES, the first column of R
is
\begin{equation*}
    \begin{split}
        & R[:,0] \\
        = & MC([S(a_{0,0}), S(a_{1,1}), S(a_{2,2}), S(a_{3,3})]) \\
        = &  MC([S(a_{0,0}), 0, 0, 0] \\
        & \ \ \ \oplus [0, S(a_{1,1}), 0, 0] \\
        & \ \ \ \oplus [0, 0, S(a_{2,2}), 0] \\
        & \ \ \ \oplus [0, 0, 0, S(a_{3,3})]) \\
        = & MC([S(a_{0,0}), 0, 0, 0]) \\
        & \oplus MC([0, S(a_{1,1}, 0, 0)]) \\
        & \oplus MC([0, 0, S(a_{2,2}), 0]) \\
        & \oplus MC([0, 0, 0, S(a_{3,3})]) \\
        = & T_0(a_{0,0}) \oplus T_1(a_{1,1}) \oplus T_2(a_{2,2}) \oplus T_3(a_{3,3})
    \end{split}
\end{equation*}

Similarly, the rest of each column are
\begin{equation*}
    \begin{split}
        & R[:,1] = T_0(a_{0,1}) \oplus T_1(a_{1,2}) \oplus T_2(a_{2,3}) \oplus T_3(a_{3,0}) \\
        & R[:,2] = T_0(a_{0,2}) \oplus T_1(a_{1,3}) \oplus T_2(a_{2,0}) \oplus T_3(a_{3,1}) \\
        & R[:,3] =T_0(a_{0,3}) \oplus T_1(a_{1,0}) \oplus T_2(a_{2,1}) \oplus T_3(a_{3,2})
    \end{split}
\end{equation*}

\subsection*{Implementation}

For S-box, we use a table that has 256 entries, each containing
the S-box output for the respective input byte.

Besides the S-box operation, we also have a MixColumns operation in the $T$ function,
so for each of $T_0, T_1, T_2, T_3$, we have a table that has 256 entries,
mapping from the input byte, i.e. the result of $S(a)$, to the output 4-byte column.
Then each $T$ function is a composition of the two table lookups.

Finally we apply the XOR operations shown in the above equations and in the
$AddRoundkey$ operation.

The memory usage of the tables is $256 + 4 * 4 * 256 = 4352$ bytes.

\end{document}
