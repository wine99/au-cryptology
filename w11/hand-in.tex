\documentclass{article}
\usepackage[left=1in,right=1in,bottom=0.8in,top=0.8in]{geometry}
\geometry{a4paper}
\usepackage[parfill]{parskip}
\usepackage{graphicx}
\usepackage{xcolor}
\usepackage{hyperref}
\usepackage{amsmath,amssymb}
\usepackage{listings}
\usepackage[outputdir=build]{minted}

\title{Cryptology Exercise Week 11}
\author{Zijun Yu 202203581}
\date{Octobor 2023}

\begin{document}

\maketitle

\section*{Hash functions from Factoring}

% Exercise 11.1 (Hash functions from Factoring) Given an RSA modulus n = pq, where p = 2p′ + 1, q = 2q′ + 1 and p′, q′ are also primes. Show that there exists elements in Z∗ n of order 2p′q′. Hint: remember that by the Chinese remainder theorem Z∗ n is isomorphic to the direct product Z∗ p × Z∗ q, and both of those are cyclic. Let g be an element of order 2p′q′, and define a function h : {0, 1}∗ → Z∗ n by h(m) = gm mod n. Show that, given a collision for h, one easily factor n. Hint: recall that we already know from Lemma 7.9 that given any multiple of (p − 1)(q − 1), we can easily factor.

We first prove that there exists an element in $Z_n^*$ of order $2p'q'$.

Let $\alpha$ and $\beta$ be a generator of $Z_p^*$ and $Z_q^*$ respectively, i.e.
$2p'$ is the smallest $i$ that satisfies $\alpha^i \mod p = 1$ and
$2q'$ is the smallest $i$ that satisfies $\beta^i \mod q = 1$.

By the Chinese remainder theorem, $Z_n^*$ is isomorphic to $Z_p^* \times Z_q^*$, $Z_p^*$.
Let $f$ be the isomorphism from $Z_n^*$ to $Z_p^* \times Z_q^*$, we prove that
the order of $g = f^{-1}(\alpha, \beta)$ is $2p'q'$.
\[
\begin{matrix}
    (&\alpha &, & \beta & ) & \leftrightarrow & g \\
    (&\alpha^2 &, & \beta^2 & ) & \leftrightarrow & g^2 \\
    (& ... &, & ... &)& \leftrightarrow & ...\\
    (& \alpha^{2p'q'} &,& \beta^{2p'q'} & )& \leftrightarrow & g^{2p'q'}
\end{matrix}
\]
We are going to prove that on the right side, $g^{2p'q'}$ is the first one that is equal to 1.
By Chinese remainder theorem, it is equivalent to prove that on the left side,
$(\alpha^{2p'q'}, \beta^{2p'q'})$ is the first pair that is equal to $(1, 1)$, which is true
because $\alpha^{2p'} \equiv 1 \mod p$ and $\beta^{2q'} \equiv 1 \mod q$ and $2p'q'$ is the least common multiple of $2p'$ and $2q'$.

Then we show that given a collision for $h$ defined by $h(m) = g^m \mod n$, one easily factor $n$.

Let $m_1$ and $m_2$ be two messages such that $h(m_1) = h(m_2)$, then
$g^{m_1} \equiv g^{m_2} \mod n$, which is equivalent to
$g^{m_1 - m_2} \equiv 1 \mod n$. This means that $m_1 - m_2$ is a multiple of $2p'q'$.
Since $(p-1)(q-1) = 4p'q'$, by multiplying $m_1 - m_2$ by any even number, we get a multiple of
$(p-1)(q-1)$. Then by Lemma 7.9, we can easily factor $n$.

\end{document}
