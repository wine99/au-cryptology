\documentclass{article}
\usepackage{graphicx} % Required for inserting images
\usepackage{listings}
\usepackage[outputdir=build]{minted}
\usepackage{amssymb,amsmath}
\usepackage[margin=1.2in]{geometry}

\title{Cryptology Exercise Week 4}
\author{Zijun Yu 202203581}
\date{September 2023}

\begin{document}

\maketitle

\section*{Exercise 6.2}

\subsection*{Part 1}

Let's first look at the $f$-function. Because the function $E$ simply
does permutation on its input (with some bits duplicated), we have
$E(\overline{R}) = \overline{E(R)}$. So we further have that
$$
    f(\overline{R}, \overline{K}) = P(S(\overline{K} \oplus E(\overline{R})))
    = P(S(\overline{K} \oplus \overline{E(R)})) = P(S(K \oplus E(R))) = f(R,K)
$$

Now let's look at each round of DES (except the last round), we have
$K'_i = \overline{K_i}$, $L'_0 = \overline{L_0}$ and $R'_0 = \overline{R_0}$.
Now, suppose that we have
$R'_{i-1} = \overline{R_{i-1}}$ and $L'_{i-1} = \overline{L_{i-1}}$,
for $i>1$, then we will have
$$L'_i = R'_{i-1} = \overline{R_{i-1}} = \overline{L_i}$$
\begin{equation*}
    \begin{split}
        R'_i & = L'_{i-1} \oplus f(R'_{i-1}, K'_i) \\
        & = \overline{L_{i-1}} \oplus f(\overline{R_{i-1}}, \overline{K_i}) \\
        & = \overline{L_{i-1}} \oplus f(R_{i-1}, K_i) \\
        & = \overline{L_i \oplus f(R_{i-1}, K_i)} \\
        & = \overline{R_i}
    \end{split}
\end{equation*}

Thus we have an induction proof that
$$
    L'_i = \overline{L_i} \text{ and } R'_i = \overline{R_i}
$$

Then, for the last round, by applying the same reasoning, we have
$$
    R'_{n} = R'_{n-1} = \overline{R_{n-1}} = \overline{R_n}
$$
$$
    L'_{n} = L'_{n-1} \oplus f(R'_{n-1}, K'_n) = \overline{L_{n}}
$$

Therefore, we have
$$
    Y' = L'_n || R'_n = \overline{L_n} || \overline{R_n} = \overline{Y}
$$

\subsection*{Part 2}

We ask the oracle to give both $Y_1 = DES_K(X)$ and $Y_2 = DES_K(\overline{X})$.
When we are checking a key $k$, we compute $y = DES_k(X)$. If $y$
is neither $Y_1$ nor $\overline{Y_2}$, we can rule out both $k$ and $\overline{k}$.
If $y = Y_1$, then $K=k$, and if $y=\overline{Y_2}$, then $K=\overline{k}$.

\end{document}
