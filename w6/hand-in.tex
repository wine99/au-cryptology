\documentclass{article}
\usepackage[left=1in,right=1in,bottom=0.8in,top=0.8in]{geometry}
\geometry{a4paper}
\usepackage[parfill]{parskip}
\usepackage{graphicx}
\usepackage{xcolor}
\usepackage{hyperref}
\usepackage{amsmath,amssymb}
\usepackage{listings}
\usepackage[outputdir=build]{minted}

\title{Cryptology Exercise Week 6}
\author{Zijun Yu 202203581}
\date{Octobor 2023}

\begin{document}

\maketitle

\section*{RSA decryption works on the entire domain}

Because $n$ is the product of two co-prime numbers $p$ and $q$,
the Chinese Remainder Theorem applies here. Because the $f$-function
in the Chinese Remainder Theorem is injective, in order to show
$x^{ed} \equiv x \mod n$, it suffices to show
that $x^{ed} \equiv x \mod p$ and $x^{ed} \equiv x \mod q$, for all $x \in  \mathbb{Z}_n$.

Here, we are showing the case where $x \notin \mathbb{Z}_n^*$. Since $n$ is the
product of the prime numbers $p$ and $q$, we have that $x$ is either
the product of $p$ and $q$, or a multiple of $p$ or $q$.

In the case where $x = p \cdot q$, we have that $x \equiv x^{ed} \equiv 0 \mod p$ and
$x \equiv x^{ed} \equiv 0 \mod q$.

We then discuss the case where $x$ is a multiple of $p$ or $q$. Without loss of generality,
we assume that $x$ is a multiple of $p$, but not of $q$.
It is obvious that $x \equiv x^{ed} \equiv 0 \mod p$.
Because $q$ is a prime number and $x$ is not a multiple of $q$, we have that
$x$ and $q$ are co-prime and $(x \text{ mod } q) \in \mathbb{Z}_q^*$. Notice that
$|\mathbb{Z}_q^*| = q-1$, hence we have
$$ (x \text{ mod } q)^{ed} \equiv x^{ed} \equiv x^{ed \text{ mod } (q-1)} \mod q$$
Since $ed \equiv 1 \mod (p-1)(q-1)$, we have that $ed \equiv 1 \mod (q-1)$, hence
$$ x^{ed} \equiv x \mod q$$

\end{document}
